\documentclass[letterpaper,11pt]{article}

\usepackage{latexsym}
\usepackage[empty]{fullpage}
\usepackage{titlesec}
\usepackage{marvosym}
\usepackage[usenames,dvipsnames]{color}
\usepackage{verbatim}
\usepackage{enumitem}
\usepackage[hidelinks]{hyperref}
\usepackage{fancyhdr}
\usepackage[french]{babel}
\usepackage{tabularx}
\usepackage{multicol}
\input{glyphtounicode}

\usepackage{baskervillef}
\usepackage[T1]{fontenc}

\pagestyle{fancy}
\fancyhf{} 
\fancyfoot{}
\setlength{\footskip}{10pt}
\renewcommand{\headrulewidth}{0pt}
\renewcommand{\footrulewidth}{0pt}

\addtolength{\oddsidemargin}{0.0in}
\addtolength{\evensidemargin}{0.0in}
\addtolength{\textwidth}{0.0in}
\addtolength{\topmargin}{0.2in}
\addtolength{\textheight}{0.0in}

\urlstyle{same}

\raggedright
\setlength{\tabcolsep}{0in}

\titleformat{\section}{
  \it\vspace{3pt}
}{}{0em}{}[\color{black}\titlerule\vspace{-5pt}]

\pdfgentounicode=1

\newcommand{\resumeItem}[1]{
  \item{
    {#1 \vspace{-4pt}}
  }
}

\newcommand{\resumeSubheading}[4]{
  \vspace{-2pt}\item
    \begin{tabular*}{0.97\textwidth}[t]{l@{\extracolsep{\fill}}r}
      \textbf{#1} & #2 \\
      \textit{\small #3} & \textit{\small #4} \\
    \end{tabular*}\vspace{-10pt}
}

\newcommand{\resumeSubItem}[1]{\resumeItem{#1}\vspace{-3pt}}
\renewcommand\labelitemii{$\vcenter{\hbox{\tiny$\bullet$}}$}
\newcommand{\resumeSubHeadingListStart}{\begin{itemize}[leftmargin=0.15in, label={}]}
\newcommand{\resumeSubHeadingListEnd}{\end{itemize}}
\newcommand{\resumeItemListStart}{\begin{itemize}}
\newcommand{\resumeItemListEnd}{\end{itemize}\vspace{-2pt}}

\begin{document}

\begin{center}
    {\LARGE Pietro Gazzi} \\ \vspace{0pt} 
    \begin{multicols}{2}
    \begin{flushleft}
    \large{93 Rue de la Roquette} \\
    \large{Paris, 75011 } \\
    \end{flushleft}
    
    \begin{flushright}
    \href{mailto:pietrogazzi01@gmail.com}{\large{pietrogazzi01@gmail.com}}\\
    \large{+330745108257}
    \end{flushright}
    \end{multicols}
\end{center}

\section{Résumé Professionnel}
\resumeSubHeadingListStart
\item{
    Ingénieur en Aérospatial dynamique avec une vaste expérience en analyse de données, modélisation numérique et gestion des risques. Un solide parcours de recherche et dans l'industrie, avec un accent fort sur l'utilisation des données pour prendre des décisions et optimiser les performances. Maîtrise de plusieurs langages de programmation et expérience dans l'utilisation d'outils d'analyse avancés.
}
\resumeSubHeadingListEnd

\section{Formation}
\resumeSubHeadingListStart
    \resumeSubheading
        {Politecnico di Milano}{Septembre 2020 -- Décembre 2022}
        {Ingénierie Aérospatiale}{}
    \resumeSubheading
        {Università degli Studi di Firenze}{Septembre 2017 -- Juillet 2020}
        {Ingénierie Mécanique}{}
\resumeSubHeadingListEnd

\section{Expérience de Recherche}
\resumeSubHeadingListStart
    \resumeSubheading
        {Composites PEKK/fibre de carbone : Mode I statique et fatigue}{Mars 2022 -- Décembre 2022}
        {Politecnico di Milano}{Milan, Italie}
      \resumeItemListStart
        \resumeItem{Conception et fabrication des échantillons de composites}
        \resumeItem{Tests en laboratoire de composites à matrice thermoplastique dans des conditions statiques et de fatigue}
        \resumeItem{Modélisation et analyse statiques et de fatigue non linéaires par FEM}
        \resumeItem{Ajustement des paramètres via l'algorithme de Nelder-Mead}
    \resumeItemListEnd
\resumeSubHeadingListEnd

\section{Expérience Professionnelle} 
\resumeSubHeadingListStart
    \resumeSubheading
      {Ingénieur de Recherche}{Février 2023 -- Présent}
      {Institut Polytechnique de Paris}{Paris, France}
      \resumeItemListStart
        \resumeItem{Analyse expérimentale et numérique de la propagation des fissures dans le polycarbonate imprimé en 3D}
        \resumeItem{Fabrication additive d'échantillons optimisés}
        \resumeItem{Évaluation numérique du comportement de fracture}
    \resumeItemListEnd

    \resumeSubheading
      {Analyste Structurel}{Mai 2022 -- Janvier 2023}
      {Thales Alenia Space, Akka Technologies}{Turin, Italie}
      \resumeItemListStart
        \resumeItem{Modélisation FEM de la structure primaire et secondaire d’un vaisseau spatial}
        \resumeItem{Analyse statique linéaire, modale, de flambage et de réponse en fréquence}
        \resumeItem{Post-traitement et analyse des contraintes avec la rédaction des rapports de contrainte}
    \resumeItemListEnd
\resumeSubHeadingListEnd

\section{Compétences Spécialisées}
\resumeSubHeadingListStart
    \resumeItem{
     \textbf{Langages de programmation :} Python, R, Matlab, Bash
    }
    \resumeItem{
     \textbf{Solveur FEM commercial :} Abaqus, Nastran, Hypermesh
    }
    \resumeItem{
     \textbf{Analyse de données :} SQL, SAS, R, SPSS, Python
    }
    \resumeItem{
     \textbf{Visualisation :} Tableau, PowerBI
    }
\resumeSubHeadingListEnd

\section{Langues}
\resumeSubHeadingListStart
    \resumeItem{
     \textbf{Italien :} Langue maternelle
    }
    \resumeItem{
     \textbf{Anglais :} Courant
    }
    \resumeItem{
     \textbf{Français :} Courant
    }
\resumeSubHeadingListEnd

\end{document}
