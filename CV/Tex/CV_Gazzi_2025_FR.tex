\documentclass[letterpaper,10.5pt]{article}

\usepackage{latexsym, fullpage, titlesec, marvosym, color, enumitem, hyperref, fancyhdr, babel, tabularx, multicol}
\usepackage[T1]{fontenc}
\usepackage{baskervillef}

% Optimisation des marges
\addtolength{\oddsidemargin}{-0.4in}
\addtolength{\evensidemargin}{-0.4in}
\addtolength{\textwidth}{0.8in}
\addtolength{\topmargin}{-0.8in}
\addtolength{\textheight}{1.2in}

\pagestyle{fancy}
\fancyhf{}
\renewcommand{\headrulewidth}{0pt}
\renewcommand{\footrulewidth}{0pt}

% Formatage des sections
\titleformat{\section}{\large\bfseries}{}{0em}{}[\color{black}\titlerule]

% Commandes personnalisées
\newcommand{\resumeEntry}[4]{
  \item\textbf{#1} \hfill #2\\
  \textit{#3} \hfill \textit{#4}
}

\newcommand{\resumeDescription}[1]{
  \vspace{-3pt}\begin{itemize}[leftmargin=0.2in]
    #1
  \end{itemize}
}

\begin{document}

% En-tête

\begin{center}
    {\LARGE Pietro Gazzi}\\
        \begin{multicols}{2}
        \begin{flushleft}
            \textbf{Paris, 75011}\\
            \href{https://pietrowei.github.io/Portfolio}{\underline{\textcolor{blue}{Mon Portfolio}}}
        \end{flushleft}
        \begin{flushright}
            \href{mailto:pietrogazzi01@gmail.com}{pietrogazzi01@gmail.com}\\
            +33 0766968051
        \end{flushright}
    \end{multicols}
\end{center}

% Résumé Professionnel
\section*{Résumé Professionnel}
Professionnel analytique doté de solides compétences en automatisation intelligente (RPA, Process Mining, OCR) et en programmation (Python, SQL, Power Platform). Expérimenté dans l'analyse des processus, la transformation numérique et l'optimisation des workflows, avec une expertise dans la mise en œuvre de solutions innovantes pour réduire les coûts et améliorer la qualité et la rapidité des livraisons. Visitez mon \href{https://pietrowei.github.io/Portfolio}{\underline{\textcolor{blue}{portfolio}}} pour plus de détails.

% Expérience Professionnelle
\section*{Expérience Professionnelle}\vspace{-5pt}
\begin{itemize}[leftmargin=0.2in]
    \resumeEntry{Analyste de Données}{Leonardo Helicopters, Milan}{Mai 2024 -- Présent}{}
    \resumeDescription{
        \item Identification des processus candidats à l'automatisation à l'aide de RPA et d'analyses de processus.
        \item Développement de solutions d'automatisation basées sur SQL et Power BI, réduisant les coûts et augmentant l'efficacité.
        \item Collaboration avec des parties prenantes pour documenter les exigences métiers et prioriser les roadmaps.
        \item Mise en œuvre de tableaux de bord et de workflows automatisés pour améliorer la conformité et la rapidité des livraisons.
    }
    \resumeEntry{Ingénieur de Recherche}{Institut Polytechnique de Paris, Paris}{Fév 2023 -- Mai 2024}{}
    \resumeDescription{
        \item Analyse des données expérimentales et numériques pour optimiser les processus de fabrication additive.
        \item Développement de scripts Python pour automatiser les analyses et améliorer les conceptions expérimentales.
    }
    \resumeEntry{Ingénieur analyste}{Thales Alenia Space, Turin}{Mai 2022 -- Jan 2023}{}
    \resumeDescription{
        \item Développement de modèles prédictifs pour optimiser les performances structurelles et réduire les coûts.
        \item Utilisation de Python et SQL pour analyser les processus et identifier des opportunités d'automatisation.
    }
\end{itemize}

% Éducation
\section*{Éducation}
\begin{itemize}[leftmargin=0.2in]
    \resumeEntry{Politecnico di Milano}{Master en Ingénierie industriel}{Sept 2020 -- Déc 2022}{}
    \resumeEntry{Università degli Studi di Firenze}{Ingénierie Mécanique}{Sept 2017 -- Juil 2020}{}
    \resumeEntry{Certifications Coursera}{}{2024-2025}{}
\end{itemize}
\begin{multicols}{2}
    \small
    \begin{itemize}[leftmargin=0.4 in, label={-}]
        \item Gestion des Investissements avec Python et Apprentissage Automatique (EDHEC)
        \item Certificat Professionnel en Analyse de Données (Google)
    \end{itemize}
    \begin{itemize}[leftmargin=0.3 in, label={-}]
        \item Spécialisation en Science des Données (Université Johns Hopkins)
        \item PostgreSQL pour Tous (Université du Michigan)
    \end{itemize}
\end{multicols}

% Compétences Spécialisées
\section*{Compétences Spécialisées}\vspace{-15pt}
\begin{multicols}{2}
\begin{itemize}[leftmargin=0.2in]
    \item \textbf{Automatisation Intelligente :} RPA (UiPath, Blue Prism), Process Mining (Celonis), OCR, Power Platform
    \item \textbf{Traitement des Données :} Nettoyage des Données, ETL, Data Wrangling
    \item \textbf{Programmation :} SQL (PostgreSQL, MySQL), Python (Pandas, NumPy, Matplotlib, Seaborn), LaTeX, HTML
    \item \textbf{Analyse Statistique :} Tests A/B, Analyse de Régression, Métriques d'Affaires, Prévision des Séries Temporelles
    \item \textbf{Gestion de Projet :} Méthodologies Agile, Documentation des Exigences, User Stories
\end{itemize}
\end{multicols}

% Langues
\section*{Langues}\vspace{-5pt}
\textbf{Italien :} Langue maternelle \hspace{20pt} \textbf{Anglais :} Courant \hspace{20pt} \textbf{Français :} Courant \hspace{20pt} \textbf{Allemand :} Débutant

\end{document}