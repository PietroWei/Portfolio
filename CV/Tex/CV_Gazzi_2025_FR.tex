\documentclass[letterpaper,10.5pt]{article}

\usepackage{latexsym, fullpage, titlesec, marvosym, color, enumitem, hyperref, fancyhdr, babel, tabularx, multicol}
\usepackage[T1]{fontenc}
\usepackage{baskervillef}

% Optimisation des marges
\addtolength{\oddsidemargin}{-0.4in}
\addtolength{\evensidemargin}{-0.4in}
\addtolength{\textwidth}{0.8in}
\addtolength{\topmargin}{-0.8in}
\addtolength{\textheight}{1.2in}

\pagestyle{fancy}
\fancyhf{}
\renewcommand{\headrulewidth}{0pt}
\renewcommand{\footrulewidth}{0pt}

% Formatage des sections
\titleformat{\section}{\large\bfseries}{}{0em}{}[\color{black}\titlerule]

% Commandes personnalisées
\newcommand{\resumeEntry}[4]{
  \item\textbf{#1} \hfill #2\\
  \textit{#3} \hfill \textit{#4}
}

\newcommand{\resumeDescription}[1]{
  \vspace{-3pt}\begin{itemize}[leftmargin=0.2in]
    #1
  \end{itemize}
}

\begin{document}

% En-tête

\begin{center}
    {\LARGE Pietro Gazzi}\\
        \begin{multicols}{2}
        \begin{flushleft}
            \textbf{Paris, 75011}\\
            \href{https://pietrowei.github.io/Portfolio}{\underline{\textcolor{blue}{Mon Portfolio}}}
        \end{flushleft}
        \begin{flushright}
            \href{mailto:pietrogazzi01@gmail.com}{pietrogazzi01@gmail.com}\\
            +33 0766968051
        \end{flushright}
    \end{multicols}
\end{center}

% Résumé Professionnel
\section*{Résumé Professionnel}
Professionnel analytique doté de solides compétences en SQL, Python et en visualisation de données (Power BI, Tableau). Expérimenté en analyse de données, modélisation prédictive et business intelligence, avec une expertise dans la transformation de données complexes en informations exploitables. Compétent en optimisation des processus et en collaboration interfonctionnelle pour soutenir les décisions stratégiques et améliorer la performance opérationnelle. Visitez mon \href{https://pietrowei.github.io/Portfolio}{\underline{\textcolor{blue}{portfolio}}} pour plus de détails.

% Expérience Professionnelle
\section*{Expérience Professionnelle}\vspace{-5pt}
\begin{itemize}[leftmargin=0.2in]
    \resumeEntry{Analyste de Maintenance}{Leonardo Helicopters, Milan}{Mai 2024 -- Présent}{}
    \resumeDescription{
        \item Analyse et réparation des composants structurels en utilisant des méthodologies basées sur les données.
        \item Développement de tableaux de bord basés sur SQL dans Power BI, réduisant le temps de réparation de 15\% et augmentant l'efficacité opérationnelle.
        \item Utilisation de l'analyse statistique et des techniques d'apprentissage automatique pour identifier les schémas de défaillance.
        \item Collaboration avec des équipes interfonctionnelles pour améliorer les processus de collecte et de rapport des données.
        \item Réalisation d'analyses de marché pour identifier les opportunités commerciales et optimiser les services de réparation.
    }
    \resumeEntry{Ingénieur de Recherche}{Institut Polytechnique de Paris, Paris}{Fév 2023 -- Mai 2024}{}
    \resumeDescription{
        \item Analyse expérimentale et numérique de la propagation des fissures dans le polycarbonate imprimé en 3D et fabrication additive de spécimens optimisés.
        \item Réalisation d'analyses de données pour soutenir les projets de recherche et améliorer les conceptions expérimentales.
    }
    \resumeEntry{Ingénieur analyste}{Thales Alenia Space, Turin}{Mai 2022 -- Jan 2023}{}
    \resumeDescription{
        \item Développement de modèles basés sur les données pour prédire les performances structurelles et optimiser les conceptions.
        \item Analyse des compromis coût-performance structurelle en utilisant Python (Pandas, NumPy), optimisant la conception des composants et réduisant les coûts de production de 5\%.
    }
\end{itemize}

% Éducation
\section*{Éducation}
\begin{itemize}[leftmargin=0.2in]
    \resumeEntry{Politecnico di Milano}{Master en Ingénierie industriel}{Sept 2020 -- Déc 2022}{}
    \resumeEntry{Università degli Studi di Firenze}{Ingénierie Mécanique}{Sept 2017 -- Juil 2020}{}
    \resumeEntry{Certifications Coursera}{}{2024-2025}{}
\end{itemize}
\begin{multicols}{2}
    \small
    \begin{itemize}[leftmargin=0.4 in, label={-}]
        \item Gestion des Investissements avec Python et Apprentissage Automatique (EDHEC)
        \item Certificat Professionnel en Analyse de Données (Google)
    \end{itemize}
    \begin{itemize}[leftmargin=0.3 in, label={-}]
        \item Spécialisation en Science des Données (Université Johns Hopkins)
        \item PostgreSQL pour Tous (Université du Michigan)
    \end{itemize}
\end{multicols}

% Compétences Spécialisées
\section*{Compétences Spécialisées}\vspace{-15pt}
\begin{multicols}{2}
\begin{itemize}[leftmargin=0.2in]
    \item \textbf{Traitement des Données :} Nettoyage des Données, ETL, Data Wrangling
    \item \textbf{Intelligence d'Affaires :} Power BI, Tableau, Excel (Avancé)
    \item \textbf{Programmation :} SQL (PostgreSQL, MySQL), Python (Pandas, NumPy, Matplotlib, Seaborn), LaTeX, HTML
    \item \textbf{Analyse Statistique :} Tests A/B, Analyse de Régression, Métriques d'Affaires, Prévision des Séries Temporelles
\end{itemize}
\end{multicols}

% Langues
\section*{Langues}\vspace{-5pt}
\textbf{Italien :} Langue maternelle \hspace{20pt} \textbf{Anglais :} Courant \hspace{20pt} \textbf{Français :} Courant \hspace{20pt} \textbf{Allemand :} Débutant

\end{document}